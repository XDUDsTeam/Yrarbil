%%%%%%%%%%%%%%%%%%%%%%%%%%%%%%%%%
% 数据结构大作业 架构设计报告
%%%%%%%%%%%%%%%%%%%%%%%%%%%%%%%%%%

%使用 XeLaTeX 和 XeCJK 套装 编译
\documentclass[UTF8]{report}
\usepackage{xeCJK}

%flag 设置
\makeatletter
\def\@NoStyleChapter{}
\makeatother



%%%%%%%%%%%%%%%%%%%%%%
% 数据结构大作业 报告导言区
%%%%%%%%%%%%%%%%%%%%%%



% 使用 color 宏包
\makeatletter
\ifdefined\@NoPagckageColor %! 如果定义了 ,不包含宏包color
\relax
\else
\usepackage{color}
\fi
\makeatother



% 对链接等的处理
\makeatletter
\ifdefined\@NoPackageHyperref
\relax
\else\usepackage[colorlinks,linkcolor=blue,anchorcolor=blue,citecolor=red,bookmarksnumbered]{hyperref}
\fi
\makeatother


% 使用 代码环境的宏包
\makeatletter
\ifdefined\@NoPackageListings
\relax
\else
\usepackage{listings}
\usepackage{xcolor}
\lstset{breaklines}
\lstset{basicstyle=\sffamily,keywordstyle=\bfseries,commentstyle=\rmfamily\itshape,escapechar='}
\lstset{flexiblecolumns}
\fi
\makeatother


%调整样式
%修改页眉与页脚
\usepackage{fancyhdr}
\pagestyle{fancy}



%章节的
\usepackage{titlesec}
%目录的
\usepackage{titletoc}



%修改Chapter的格式
\makeatletter
\ifdefined\@NoStyleChapter
\relax
\else\titleformat{\chapter}
    [display]
    {\centering\Huge\bfseries}
    {第\,\thechapter\,部分}
    {1em}
    {}
\fi
%修改Chapter的目录格式
\titlecontents{chapter}
[0pt]
{\addvspace{2pt}\filright}
{\contentspush{}}%\thecontentslabel\ }}
{a}
{\titlerule*[8pt]{.}\contentspage}
\makeatother

%定义 BibTeX 图标
\def\BibTeX{{\rm B\kern-.05em{\sc i\kern-.025em b}\kern-.08em
		T\kern-.1667em\lower.7ex\hbox{E}\kern-.125emX}}


%对于 api 版本的控制
\makeatletter
\ifx \apiver \@empty
\def\apiver{!!}
\fi
\ifdefined\@APIVersionFlag
\def\Color@now#1{\colorbox[rgb]{0.98,0.72,0.43}{#1}}
\def\Color@old#1{\colorbox[rgb]{0.79,0.79,0.58}{#1}}
\def\APIColor#1#2{
    \def\tmp@a{#1}
    \def\tmp@b{#2}
    \ifx \tmp@a \tmp@b
    \Color@new{#1}
    \else
    \Color@old{#1}
    \fi
    }
\def\apiversionn#1{\paragraph{\APIColor{\small#1}{\apiver}}}
\def\apiversion#1#2{\paragraph{{\colorbox[rgb]{0.98,0.72,0.43}{#1}} #2}\stepcounter{paragraph}}
\else
\def\apiversion#1{#1}
\def\apiversionn#1{#1}
\fi
\makeatother




%对于附录的设置
\makeatletter
\ifdefined\@UsingAppendix
\usepackage[titletoc]{appendix}
\renewcommand\appendixname{附录}
\renewcommand\appendixpagename{附录}
\appendixtitleon
\else
\relax
\fi
\makeatother

% 设置HTTP请求参数显示的背景色
\makeatletter
\ifdefined\@RequestArgColor
\def\reqargclr#1{\item[{\colorbox[rgb]{1.00,0.90,1.00}{#1}}]}
\fi
\makeatother


% 使得脚注标号随页码更新
\makeatletter

\ifdefined\@FootnoteWithPage
\usepackage{chngcntr}
\counterwithin{paragraph}{section}
\counterwithin{paragraph}{subsection}
\counterwithin{paragraph}{subsubsection}
\counterwithin{footnote}{paragraph}
\renewcommand\thefootnote{\theparagraph-\alph{footnote}}
\fi
\makeatother


% 定义 JSON
\makeatletter
\ifdefined\@NoPackageListings
\relax
\else
\ifdefined\@lst@json@define
\lstdefinelanguage{JSON}
{
    keywords={true,false,null}
}
\fi
\lstdefinelanguage{JavaScript}%
{
    morekeywords={true,false,null},%
    alsoletter={:},
    moredelim=[s][{\color[rgb]{0.67,0.00,0.67}}]{"}{"},
    moredelim=[s][{\color[rgb]{0.36,0.67,0.78}}]{:"}{"},%
    identifierstyle=\color{blue},
    emph={:}, emphstyle=\color{red}
}[keywords,comments,strings]%
\fi
\makeatother

%%%%%%%%
% 数据结构大作业 报告作者 
%%%%%%%%


\author{李约瀚 \\ qinka@live.com \\ 14130140331 
    \and 褚欣 \\ m15949075919@163.com \\14130140356
    \and 戚瑶 \\ 631987611@qq.com \\14130140362
    \and 乔新文 \\ starsriver@outlook.com \\14130140393  
    \and 殷熔磾 \\ yinrongdi@163.com \\ 14050120069 }

\title{架构设计报告}
\begin{document}
    \maketitle
    
%%%%%%%%%%%%%%%%%%%%%%%%%
% 数据结构大作业 项目介绍
%%%%%%%%%%%%%%%%%%%%%%%%%

\section*{项目介绍}
我们的图书管理系统的名称叫做 \href{http://yrarbil.iok.la}{Yrarbil} 是 图书馆英文单词 \textbf{library} 的逆写。\footnotetext{我们喜爱开源项目与和开源相关的创业项目。}
\subsection*{团队介绍}
XDUDsTeam 开发团队由 西安电子科技大学软件学院1413014班的五名同学组成。他们依次是:
\paragraph{李约瀚} 李约瀚就读于 西安电子科技大学软件学院软件工程专业,目前是一名大二学生。
主要使用 Haskell C\# C++ 等语言进行开发。
\subsection*{项目介绍}
该项目是一个简单地图书馆图书管理系统。可用于小型图书馆的图书管理工作。

该项目是起于西安电子科技大学软件学院2015年数据结构课程大作业。目前遵循 BSD3 开原协议 \footnote{有计划更换开原协议。}。这个项目后端使用以 \href{https://www.haskell.org}{Haskell} 编写的框架 wrap 与 wai \footnote{其中主要是用到了一个名为 \href{https://www.yesodweb.com}{Yesod} 的框架}。同时使用著名的开源数据库
\href{http://www.postgresql.org}{PostgreSQL} ,作为存储数据的方式。
整个项目使用 \href{https://travis-ci.org}{Travis-ci} 持续集成服务,与 \href{https://www.docker.com/}{Docker} 容器服务。同时计划在DaoCloud \footnote{道客云,国内的Docker服务创业公司。}
\subsection*{项目信息}
这个项目是遵行 BSD3 开源协议开源的。托管于 GitHub 上的   \href{https://github.com/XDUDsTeam/}{Repo}。
我们计划将把稳定的编译后的二进制程序通过 Docker 镜像直接发布于
\href{https://hub.docker.com/}{Docker Hub}
\footnote{Docker Hub 可能会被GFW封锁。\tiny{至于原因,帝国主义反动势力在作祟。}如果你在XDU的校内网访问的话,可能也无法访问。上述GFW的制作者在XDU演讲时当场打脸。}
上。
    \pdfbookmark[1]{text}{anchor}
    \tableofcontents
    \part{前端架构}
    前端负责域用户的交互等行为。
    \chapter{桌面应用部分}
    该部分为图书管理员处理图书的各项工作用的客户端。
    相对应的设置,请保留到本地文件中。
    \section{UI/交互部分}
    \subsection{管理员登陆部分}
    管理员使用用户名与密码
    \footnote{改密码请遵寻UNIX类密码}
    登入管理系统的界面。
    \subsection{管理员退出部分}
    管理员确认退出的界面。
    \subsection{图书借阅部分}
    在该部分,图书管理员先输入读者条码,然后再输入读者所借阅的图书的条码。
    当一个读者借阅完之后,读者条码那一部分需要复位。
    \subsection{图书归还部分}
    这个步骤无需借阅着的条码。直接输入读者的条码即可。
    如果有超期\footnote{应该有系统判断是否超期。}、污损\footnote{该部分应该有图书管理员负责。}的情况
    \subsection{图书续借部分}
    直接输入条码即可,然而续借次数有限制。
    \subsection{借阅处罚部分}
    对于各种原因的罚款,进行处理。记录。
    \subsection{图书录入部分}
    将一本书录入到系统中。
    \subsection{图书处理部分}
    将由于各种原因的出书处理,记录在案,同时图书销毁。
    \section{数据处理部分}
    负责各类数据处理
    \subsection{数据请求}
    将UI部分要求的数据请求进行处理,然后请求。
	\begin{description}
	\item[登录] 将UI传入的数据,制作成所需访问的URL与对应参数,同时生成HTTP
		请求。
	\end{description}
    \subsection{数据解析}
    将后端返回的数据处理之后,发回UI部分。
	\begin{description}
	\item[登录] 返回的是JSON数据,要求将JSON解析。JSON数据要么为具体内容
		\footnote{具体数据格式,详见API文档。}
		,要么数错误信息。
	\item[登出]
	\end{description}
    \subsection{数据处理}
    按照UI的需求,处理数据。
    \chapter{网页交互部分}
    基于网页的技术。
    \section{访客查询部分}
    查询现有图书信息,以及图书是否在架信息。
    \subsection{图书信息查询}
    图书的CIP信息
    \footnote{依据作者,图书题目,ISBN,,等信息查询。}
    ,图书的引索信息。提供图书的标准的 \BibTeX 数据。
    \subsection{图书状态查询}
    查询图书的在架状态,预约状态,借阅的最长期限。
    \section{读者查询部分}
    主要是指,查询并且操作。
    \subsection{预约}
    对于某一本图书的预约,可以完成的是对在架与借阅中的图书的预约。同时预约期间不可以被其他人借阅,而有一定期限。
    \subsection{续借}
    对于已经借阅的图书,进行续借,续借的次数有限制。
    \part{后端架构}
    \setcounter{chapter}{0}
    \chapter{数据库部分}
    \chapter{请求处理部分}

\end{document}
